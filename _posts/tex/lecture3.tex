\chapter{Even more universal problems}
\lecturedetails{12 October 2017}{M Fiore, S Borgeaud, R Kusztos}

In this lecture we introduce mathematical terminology and concepts to then
consider more interesting universal properties.

We begin by further analysing the definitions of joins and meets.

\section{Joins and Meets}

Consider $(P, \leq)$ to be a preorder. For any subset $S \subseteq P$ we can
define $\bigvee S$, which is known as the \emph{join}, \emph{sup} (supremum),
or \emph{lub} (least upper bound) of the subset $S$, as well as $\bigwedge S$,
the \emph{meet}, \emph{inf} (infimum) or \emph{glb} (greatest lower bound)  of
the subset $S$.

\begin{definition}[Joins]
\label{defjoins}
    For a subset $S \subseteq P$, if it exists, $\bigvee S \in P$ is the join
    of S if it satisfies the universal properties:
        \begin{enumerate}
            \item $\forall x \in S$ it must be the case that $x \leq \bigvee S$
            \item $\forall u \in P$ we have $(\forall x \in S \ldotp x \leq u)
                \Rightarrow \bigvee S \leq u$
        \end{enumerate}
\end{definition}

The concept of a meet is dual to that of a join, so the definition is obtained
by reversing the preorder.

\begin{definition}[Meets]
\label{defmeets}
    For a subset $S \subseteq P$, if it exists, $\bigwedge S \in P$ is the meet
    of S if it satisfies the universal properties:
    \begin{enumerate}
    \item $\forall x \in S$ it must be the case that $ \bigwedge S \leq x$
    \item $\forall l \in P$ we have $(\forall x \in S \ldotp l \leq x)$, $
      \Rightarrow l \leq \bigwedge S$
    \end{enumerate}
\end{definition}
For a better understanding of these universal properties, we can consider some
special cases:
\begin{itemize}
\item $S = \{a, b\}$, we can use the notations $\bigvee S = (a \vee b)$ and,
  analogously $\bigwedge S = (a \wedge b)$
\item $S = \{a\}$. In this case, we are guaranteed that the meet and join exist:
  $\bigvee S = \bigwedge S = a$
\item $S = \emptyset$. Considering the definition of joins, the first property
  becomes true due to the inability to pick an $x \in S$. With the same
  reasoning, the second property reduces to $\forall u \in P \Rightarrow \bigvee
  S \leq u$. Therefore the solution is $\bigvee S = \bot$, the bottom element in
  P. Analogously, $\bigwedge S = \top$, the top element of P.
\end{itemize}

We can relate the case where S has two elements, $S = \{a, b\}$ to previous
examples by analysing them categorically.

For $S = \{a, b\}$, we can write the universal property as a commutative diagram

\begin{center}
    \begin{tikzcd}[ampersand replacement=\&]
        {} \& a \vee b \arrow[dd, dashed, "\exists ! "] \& {} \\
        a \arrow[ru, "\leq"] \arrow[rd, swap]
        \& {}
        \&
        b \arrow[lu, swap, "\leq"] \arrow[ld]
        \\
        {} \& c \& {}
    \end{tikzcd}
\end{center}

If instead of $\leq$ we would write $\rightarrow$, we can interepret the
diagram in the world of sets and functions, obtaining the properties of the
disjoint union of sets.

\begin{center}
    \begin{tikzcd}[ampersand replacement=\&]
        {} \& A \uplus B \arrow[dd, dashed, "\exists ! "] \& {} \\
        A \arrow[ru] \arrow[rd, swap]
        \& {}
        \&
        B \arrow[lu, swap] \arrow[ld]
        \\
        {} \& C \& {}
    \end{tikzcd}
\end{center}

Since we picked an arbitrary subset of P, we can apply this reasoning to
the world of sets and functions, by introducing a universal initial solution
for the problem of adding a family of sets.

\subsection{Adding families of sets}

Given a family of sets $\setof{A_i}_{i\in I}$ \big(also denoted as
$\setof{ A_i \suchthat i \in I }$ or $(A_i\suchthat i\in I)$\big) indexed by a
set $I$, we seek:
\begin{itemize}
\item
  a set $\biguplus_{i \in I} A_i$ and a family of functions
  $\setof{ in_k: A_k\to \biguplus_{i \in I} A_i }_{k\in I}$ such that,
\item
  for every set $C$ and family of functions
  $\setof{ f_k: A_k\to C }_{k\in I}$, there is a unique $u$ such that the
  diagram below commutes for all $k\in I$:
\begin{center}
    \begin{tikzcd}[ampersand replacement=\&]
        {} \& \biguplus_{i \in I} A_i \arrow[dd, dashed, "\exists ! u"] \& {} \\
        A_k \arrow[ru, "in_k"] \arrow[rd, swap, "f_k"]
        \& {}
        \\
        {} \& C \& {}
    \end{tikzcd}
\end{center}
\end{itemize}

An example of such initial universal solution is given by
\[ \biguplus_{i \in I} A_i \,=\, \bigcup_{i \in I}\ \{i\} \times A_i \]
In which case the functions $in_k$ are given by
\[ in_k(x) = (k, x)\]

We can try to find further analogies in the case where P is a poset. Consider
the following properties of joins in a poset: $\forall a, b, c \in P \ldotp$
\begin{align*}
    a \vee b &= b \vee a \\
    (a \vee b) \vee c &= a \vee (b \vee c) \\
    a \vee a &= a \\
\end{align*}

We can try to reinterpret these equations in the world of sets and functions,
by replacing $\vee$ (join) with $\uplus$~(sum) and equality with $\cong$
(isomorphism).  We obtain:
\begin{align*}
    A \uplus B &\cong B \uplus A  \\
    (A \uplus B) \uplus C &\cong A \uplus (B \uplus C) \\
    A \uplus A &\ncong A \\
\end{align*}

Note that \emph{idempotency} is a property particular to posets, because the
arrows~(\ie, the $\leq$ relation) between elements of a poset is unique.
However, we always have:
\begin{center}
  $A \to A\uplus A$
  \quad and \quad
  $A\uplus A \to A$
  \enspace.
\end{center}

\begin{exercise}
    Construct the bijections from the universal properties. \\
    Hint: For the first property, recall that
    \begin{align*}
        A \uplus B &= (\{0\} \times A) \cup (\{1\} \times B)
    \end{align*}
    Thus we can define the bijections:
    \begin{align*}
        (0, a) &\mapsto (1, a)  \\
        (1, b) &\mapsto (0, b)
    \end{align*}
\end{exercise}

\section{Solution for Exercise \ref{ex:bug}}

Consider whether the following holds:
\begin{center}
    \begin{tikzcd}[ampersand replacement=\&]
        S \arrow[r, "x \mapsto \{x\}"] \arrow[dr, swap, "\forall\ f"] \&
        \mathcal{P}_{\mathrm{fin}}(S) \arrow[d, "\exists ! h"]
        \& (\mathcal{P}_{\mathrm{fin}}(S), \cup)
        \arrow[d, "\text{$h$homomorphism}"]
        \&
        \\
        {}
        \& L \& (L, \vee) \&
    \end{tikzcd}\\[2mm]
\end{center}
Since this diagram commutes
\begin{align*}
    h(\{x_1, x_2 ..., x_n \}) = f(x_1) \vee f(x_2) \vee ... f(x_n)
    \qquad (n\in\nats)
\\
\intertext{Since}
    \{x_1, x_2 ..., x_n \} = \bigcup_i \{x_i\}
\intertext{and}
    h\big(\bigcup_i \{x_i\}\big) = \bigvee_i h(\{x_i\})
\intertext{we get that}
    f(x_i) = h(\{x_i\})
\end{align*}

Since this is not defined for the empty set, we need restrict our $\tilde{S}$
to be $\mathcal{P}_{\mathrm{fin}}^{+}(S)$

\section{Suplattice}

\begin{definition}
A suplattice is a poset $(L, \le)$ with the property that there exists a join
${\bigvee}S$ for every subset $S \subseteq L$.
\end{definition}

Note, that as $\emptyset \subseteq L$, a suplattice always has a least
element.

Consider the following problem,
\begin{center}
    \begin{tikzcd}[ampersand replacement=\&]
        S \arrow[r] \arrow[dr, swap, "\forall\ f"] \&
        \tilde{S} \arrow[d, "\exists ! h"] \&
        (\tilde{S},\bigvee_{\tilde{S}}) \arrow[d, "h"] \&
        \text{a suplattice}
        \\
        \&
        L \&
        (L,\bigvee_L) \&
        \text{a suplattice}
    \end{tikzcd} \\[3mm]
    \text{where $h$ must preserve structure, \ie}
      \[
        \forall X \subseteq S\ldotp
          h(\bigvee_{\tilde{S}}X) = \bigvee_L \setof{h(x) \suchthat x \in X}
      \]
\end{center}

\begin{exercise}
    Show that for $\widetilde{S} = \mathcal{P}(S)$ with
    $\bigvee_{\widetilde{S}}=\bigcup$ the function
    $S \to \widetilde{S}: x \mapsto \setof{ x }$ is an initial solution.
\end{exercise}

\section{Two more universal problems}

\subsection{Problem 1}

For $S$ a set, consider
\begin{center}
    \begin{tikzcd}[ampersand replacement=\&]
        s \in S \arrow[r, "\sigma"] \&
        S \&
    \end{tikzcd} \\[3mm]
\end{center}

How can we compare two such gadgets $(s_1, S_1, \sigma_1)$ and $(s_2, S_2, \sigma_2)$?
We need a function $f: S_1 \to S_2$ such that
\begin{enumerate}
    \item $f(s_1) = s_2$
    \item $\forall x \in S \ldotp f(\sigma_1(x)) = \sigma_2(f(x))$
\end{enumerate}
We can express the second condition in the usual diagrammatic notation as
    \begin{center}
        \begin{tikzcd}[ampersand replacement=\&]
            S_1 \arrow[r, "f"] \arrow [d, "\sigma_1"] \&
            S_2 \arrow[d, "\sigma_2"] \&
               \\
            S_1 \arrow[r, "f"] \&
            S_2 \&
        \end{tikzcd} \\[3mm]
    \end{center}

\subsubsection{Similarity to the problem on monoids}

Note that this problem is not too different from problem
\ref{section_generate_monoid} for monoids. Recall, a monoid $(M, e, \star)$
consists of a set $M$, a neutral
element $e \in M$ and a binary operation $\star: M\times M \to M$.
Homomorphisms between $(M_1,e_1,\star_1)$ and $(M_2,e_2,\star_2)$ are
functions $h: M_1 \to M_2$ such that \[ h(e_1) = e_2 \]
and
    \[ h(x \star_1 y) = h(x) \star_2 h(y) \]
Rewriting the second condition as a diagram, the connection between the two
problems arises:
\begin{center}
    \begin{tikzcd}[ampersand replacement=\&]
        M_1 \times M_1 \arrow[r, "h \times h"] \arrow [d, "\star_1"] \&
        M_2 \times M_2 \arrow[d, "\star_2"] \&
           \\
        M_1 \arrow[r, "h"] \&
        S_2 \&
    \end{tikzcd} \\[3mm]
\end{center}
where
\[\big(h \times h\big)(x,y) = \big(h(x), h(y)\big)\]

\subsubsection{Intuition}
For the intial solution of the posed problem on sets, we need the following
\[s \in S \]
\[\sigma(s) \in S \]
Thus, as $\sigma(s) \in S$, we need to add
\[\sigma(\sigma(s)) \in S\]
and similarly, we need to add the following elements
\[\sigma(\sigma(\sigma(s))) \in S\]
\[\vdots\]
This structure suggests a connection to the natural numbers $\nats$.
\begin{exercise}
    Show that
    \begin{center}
        \begin{tikzcd}[ampersand replacement=\&]
            0 \in \nats \arrow[r, "\mathrm{succ}"] \&
            \nats \&
        \end{tikzcd} \\[3mm]
    \end{center}
    is the initial solution. That is the natural numbers $\nats$ with 0 and
    the successor function $\mathrm{succ} = \lambda x\in\nats.\ x + 1$.
\end{exercise}
\begin{proof}[Solution]
    We want to show that there is a unique $h: \nats \to S$ such that for any
    data
    \begin{center}
        \begin{tikzcd}[ampersand replacement=\&]
            s \in S \arrow[r, "\sigma"] \&
            S \&
        \end{tikzcd}
    \end{center}
    the following diagram commutes:
    \begin{center}
        \begin{tikzcd}[ampersand replacement=\&]
            0 \arrow[d, mapsto, "h"] \& \in \&
            \nats \arrow[r, "\mathrm{succ}"] \arrow[d, "h"] \&
            \nats \arrow[d, "h"] \\
            s \& \in \& S \arrow[r, "\sigma"] \& S
        \end{tikzcd}
    \end{center}
    That is, we want to have $h \comp \mathrm{succ} = \sigma \comp h$ and
    $h(0) = s$.

    Consider the map $h(n) = \sigma^n (s)$, where $\sigma^0(s)\eqdef s$ and
    $\sigma^{n+1}(s)\eqdef\sigma\big(\sigma^n(s)\big)$.  It is clear that
    $h(0) = s$; also, for every $n \in \nats$,
    \[ (h \comp \mathrm{succ})(n) = h(n+1) = \sigma^{n+1}(s) =
    \sigma(\sigma^{n}(s)) = (\sigma \comp h)(n) \]
    and therefore $h \comp \mathrm{succ} = \sigma \comp h$.

    Let us show that $h$ is unique with the above property, \ie~for any
    $k:\nats\to S$ such that $k(0) = s$ and $k \comp \mathrm{succ} = \sigma
    \comp k$ we have $k = h$, by establishing that, for all $n \in \nats$,
    $k(n) = \sigma^{n} (s)$ by induction:

    \begin{itemize}

    \item When $n=0$, we have $k(0) = s = \sigma^0(s)$ by assumption.

    \item Assume that $k(n) = \sigma^n(s)$ for $n \in \nats$, and consider
    $k(n+1)$:
    \[ k(n + 1) = (k \comp \mathrm{succ})(n) = (\sigma \comp k)(n) =
    \sigma(k(n)) = \sigma(\sigma^{n}(s)) = \sigma^{n+1}(s) \]
    Thus, $k(n + 1) = \sigma^{n + 1}(s)$.

    \end{itemize}

    By induction, we conclude $k(n) = h(n)$ for all $n\in\nats$; so that
    $k = h$, and the function is unique.  Therefore this is the initial
    solution.
\end{proof}


\subsubsection{Relation to automata theory}

This problem can be seen from an automata-theoretic point of view, $s$ being
the start state and $\sigma$ the transition function. Unfortunately, due to
time constraints, we will not be exploring this relation further in the
course.

\subsection{Problem 2}
Consider the following gadget for $S$ a set
\begin{center}
    \begin{tikzcd}[ampersand replacement=\&]
        \&
        \{0,1\} \&
        \\
        S \arrow[ur, "\theta"] \arrow[r, "\sigma"]\&
        S
    \end{tikzcd} \\[3mm]
\end{center}
We compare two such gadgets $(S_1, \theta_1, \sigma_1)$ and $(S_2, \theta_2, \sigma_2)$
with the function $h: S_1 \to S_2$ such that
\begin{enumerate}
    \item $\theta_2 \comp h = \theta_1$
    \item $\sigma_2 \comp h = h \comp \sigma_1$
\end{enumerate}
Diagrammatically, this is equivalent to
\begin{center}
    \begin{tikzcd}[ampersand replacement=\&]
        \&
        \{0,1\}
        \&
        \&
        \&
        \\
        S_1
        \arrow[ur, "\theta_1"]
        \arrow[d, swap, "\sigma_1"]
        \arrow[rr, "h"]
        \&
        \&
        S_2
        \arrow[ul, swap, "\theta_2"]
        \arrow[d, "\sigma_2"]\&
        \\
        S_1 \arrow[rr,"h"]
        \&
        \&
        S_2 \&
    \end{tikzcd} \\[3mm]
\end{center}

\subsubsection{Relation to automata theory}
Again, we can view this problem from an automata point of view. As before, $S$
is the set of states and $\sigma$ is the transition function. Additionally, we
have $\theta$ which acts as an observing (or output) function that accepts a
state $s \in S$ if $\theta(s) = 1$ and rejects it otherwise.

The mapping $h$ tells us how a state $s_1 \in S_1$ is interpreted in $S_2$.

\begin{exercise}
    Find the final solution.

    Intuition: we need to find a set $S$ with the following structure:
    \begin{center}
        \begin{tikzcd}[ampersand replacement=\&]
            \&
            0/1 \&
            0/1 \&
            0/1 \&
            \dots \&
            \\
            s \in S \arrow[ur, mapsto, "\theta"]
                    \arrow[r, mapsto, "\sigma"]\&
            \sigma(s) \in S \arrow[ur, mapsto, "\theta"]
                            \arrow[r, mapsto, "\sigma"]\&
            \sigma(\sigma((s)) \in S \arrow[ur, mapsto, "\theta"]
                                     \arrow[r, mapsto, "\sigma"]\&
            \sigma(\sigma(\sigma(s))) \in S
                                    \arrow[ur, mapsto, "\theta"]
                                    \arrow[r, mapsto, "\sigma"]\&
            \dots \&
        \end{tikzcd}
    \end{center}
\end{exercise}
\begin{proof}[Solution]
    Consider the set of all functions from natural numbers to $\setof{0,1}$,
    written here as ${[\nats\Rightarrow\setof{0,1}]}$, and also the
    functions
    \[
      \mathrm{hd}: [\nats\Rightarrow\setof{0,1}] \to \setof{0,1}
      \enspace,\quad
      \mathrm{tl}
      : [\nats\Rightarrow\setof{0,1}] \to [\nats\Rightarrow\setof{0,1}]
    \]
    defined by
    \begin{align*}
      \mathrm{hd} (f) &= f(0)
      \\[1mm]
      \mathrm{tl} (f) &= \lambda x \in \nats.\ f (x + 1)
    \end{align*}
    This clearly provides data as desired, it remains to show it is a final
    solution.

    Consider any other data with set $S$, and functions
    $\theta: S \to \setof{0, 1}$ and $\sigma: S \to S$.
    Let ${h: S \to [\nats\Rightarrow\setof{0,1}]}$ be the following function:
    \[ h (x) \eqdef \lambda n \in \nats.\ \theta\big(\sigma^n (x)\big) \]
    It satisfies the required conditions:

    \begin{itemize}

    \item
    $\theta = \mathrm{hd} \comp h$ because
    \[
      \big(\mathrm{hd} \comp h\big)(x)
      = \mathrm{hd}
          \big( \lambda n \in \nats.\ \theta\big(\sigma^n(x)\big) \big)
      = \theta\big(\sigma^0(x)\big)
      = \theta(x)
    \]

    \item
    $h \comp \sigma = \mathrm{tl} \comp h$ because
    \[
      \big(h \comp \sigma\big)(x)
      = h \big( \sigma (x) \big)
      = \lambda n \in \nats.  \, \theta\big(\sigma^n\big(\sigma(x)\big)\big)
      = \lambda n \in \nats.  \, \theta\big(\sigma^{n+1}(x)\big)
    \]
    and
    \[
      \big(\mathrm{tl} \comp h\big)(x)
      = \mathrm{tl}
          \big(\lambda m \in \nats.\, \theta\big(\sigma^m (x)\big)\big)
      = \lambda n \in \nats.\, \theta\big(\sigma^{n+1} (x)\big)
    \]

    \end{itemize}

    This function $h$ is also the unique with the above properties. To see
    this, let $k: S\to[\nats\Rightarrow\setof{0,1}]$ be any other function
    such that $\theta = \mathrm{hd} \comp k$ and
    $k \comp \sigma = \mathrm{tl} \comp k$.
    Then for any $x \in S$:
    \[
      \big(\mathrm{hd} \comp k\big) (x) =  \theta(x)
    \]
    which, by definition of $\mathrm{hd}$, implies
    \[ k(x)(0) = \theta(x) \]
    So $k(x)$ has to map $0$ to $\theta(x)$.

    Also, for $n\in\nats$, we have:
    \[
      \big(\mathrm{tl} \comp k\big) (x) (n)
      = \mathrm{tl} \big(k(x)\big) (n)
      = k(x)(n + 1)
    \]
    and, by assumption,
    \[
      \big(\mathrm{tl} \comp k\big) (x) (n)
      = \big(k \comp \sigma\big) (x) (n)
      = k \big(\sigma (x)\big)(n)
      \enspace;
    \]
    from which we conclude that $k(x)(0) = \theta(x)$ and, for all $n\in\nats$,
    $k(x)(n+1) = k\big(\sigma(x)\big)(n)$.

    Finally, one shows by induction, that
    $k(x)(n) = \theta(\sigma^n (x))$.  So that $k(x)=h(x)$ for all $x\in S$,
    and thus $k=h$.
  \end{proof}

